\documentclass[notitlepage, 12pt]{article}
% package includes
% ---------------------

\usepackage{graphicx} % for include images

\usepackage{url}
\usepackage{caption}
\usepackage{fullpage}
\usepackage{fancyhdr}
\usepackage{amsmath}
\usepackage{subcaption}
\usepackage{float}
\restylefloat{table}
\restylefloat{figure}
\usepackage{setspace}
\graphicspath{{./Pictures/}}

% \pagestyle{fancy}
% \lhead{Team 4552}
% \chead{Problem A}
% \rhead{}

% \renewcommand{\headrulewidth}{0.8pt}
% \renewcommand{\footrulewidth}{0.8pt}

\title{\textbf{INSERT TITLE HERE}}
\author{Team 4552}

\date{November 15, 2013}

\begin{document}
\maketitle

\tableofcontents
\newpage
\section{Problem Restatement}
Given a fictitious district made up of six separate zones, and average travel times between zones, create a model to optimize the placement of $n$ ambulances in order to maximize the number of civilians reached in an eight minute period. Consider the scenarios where $n = 3$, $n = 2$, and $n = 1$, respectively, and for each scenario, keep track of how many people are not being reached with each possible solution.

Then, consider a scenario in which a large scale disaster affects a single location (i.e. 9/11), and discuss how an Emergency Service Coordinator would cover the situation. Examine how a real-world city or county would prepare for such a disaster. Finally, write a two page memo detailing the model and its analysis for the Emergency Service Coordinator.


\section{Assumptions and Justifications}

\section{The Model}

\subsection{Model Approach}

\subsection{Part 1?}

\subsection{Part 2?}

\section{Model Analysis}

\subsection{3 Ambulance Cover}

\subsection{2 Ambulance Cover}

\subsection{1 Ambulance Cover}

\subsection{Catastrophic Cover}

\section{Strengths and Weaknesses}

\section{Extensions}

\section{Non-Technical Memo}

\appendix
\section{Code}





\bibliographystyle{chicago}
\bibliography{bibliography}
\end{document}
