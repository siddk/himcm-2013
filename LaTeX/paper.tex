\documentclass[notitlepage, 12pt]{article}
% package includes
% ---------------------

\usepackage{graphicx} % for include images

\usepackage{url}
\usepackage{caption}
\usepackage{fullpage}
\usepackage{fancyhdr}
\usepackage{amsmath}
\usepackage{subcaption}
\usepackage{float}
\restylefloat{table}
\restylefloat{figure}
\usepackage{setspace}
\graphicspath{{./Pictures/}}

% \pagestyle{fancy}
% \lhead{Team 4552}
% \chead{Problem A}
% \rhead{}

% \renewcommand{\headrulewidth}{0.8pt}
% \renewcommand{\footrulewidth}{0.8pt}

\title{\textbf{INSERT TITLE HERE}}
\author{Team 4552}

\date{November 15, 2013}

\begin{document}
\maketitle

\tableofcontents
\newpage
\section{Problem Restatement}
Given a fictitious district made up of six separate zones, and average travel times between zones, create a model to optimize the placement of $n$ ambulances in order to maximize the number of civilians reached in an eight minute period. Consider the scenarios where $n = 3$, $n = 2$, and $n = 1$, respectively, and for each scenario, keep track of how many people are not being reached with each possible solution.

Then, consider a scenario in which a large scale disaster affects a single location (i.e. 9/11), and discuss how an Emergency Service Coordinator would cover the situation. Examine how a real-world city or county would prepare for such a disaster. Finally, write a two page memo detailing the model and its analysis for the Emergency Service Coordinator.


\section{Assumptions and Justifications}

\section{The Model}

\subsection{Model Approach}

\subsection{Part 1?}

\subsection{Part 2?}

\section{Model Analysis}

\subsection{3 Ambulance Cover}

\subsection{2 Ambulance Cover}

\subsection{1 Ambulance Cover}

\subsection{Catastrophic Cover}
In a localized catastrophic event such as those that happened on September 11, 2001, there will be
many casualties in a single location that require aid. Ambulances, and other emergency vehicles,
such as police cars and fire trucks, will need to respond quickly in order to aid the most critically
injured first, who will be sorted by triage. Therefore, it is very important for ambulances to be able to
reach the emergency as quickly as possible, regardless of the initial location of the ambulances and the
location of the emergency.

Catastrophic events will cause more casualties than emergency personnel will be able to service, so as a result,
ambulances will have to act as shuttles to hospitals for the injured. Therefore, to maximize throughput
cities should arrange hospitals to minimize the average time between all areas and the nearest
hospital.

Since metropolitan areas are most likely to be affected by catastrophic events, and catastrophic events there
will affect significantly more people, to maximize patient throughput cities should arrange hospitals to minimize
the average time between all areas and the nearest hospital. Futhermore, due to differences in population density,
statistically, an accident in a metropolitan area will demand more urgent care than a similar accident
in a rural or suburban area. Therefore, when the city lays out its stations for ambulances, attempting to minimize
travel times to any area, it should put significantly more weight on minimizing travel time to metropolitan areas.

\section{Strengths and Weaknesses}

\section{Extensions}

\section{Non-Technical Memo}

\appendix
\section{Code}





\bibliographystyle{chicago}
\bibliography{bibliography}
\end{document}
